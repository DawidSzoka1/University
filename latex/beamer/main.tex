\documentclass[aspectratio=169]{beamer}

\usepackage[utf8]{inputenc}
\usepackage[T1]{fontenc}
\usepackage[polish]{babel}

% --- PAKIETY GRAFICZNE ---
\usepackage{tikz}
\usetikzlibrary{positioning, arrows.meta, backgrounds}
\usepackage{pgfplots}
\usepackage{listings}
\usepackage{xcolor}
\pgfplotsset{compat=1.15}

% --- KONFIGURACJA TRYBU PREZENTERA ---
\usepackage{pgfpages}
\setbeameroption{show notes on second screen=right}

\setbeamertemplate{note page}{
    \pagecolor{gray!10}
    \vskip0.5cm

% Używamy kolumn, żeby ustawić rzeczy obok siebie
    \begin{columns}[t] % [t] wyrównuje do góry
        % KOLUMNA 1: Miniatura (mała)
        \begin{column}{0.25\paperwidth}
            \centering
            \insertslideintonotes{0.25}
            \vspace{1cm}
            \tiny \textbf{Podgląd}
        \end{column}

        % KOLUMNA 2: Tekst (dużo miejsca)
        \begin{column}{0.70\paperwidth}
            \large
            \color{black}
            \textbf{Notatki:}\par
            \vspace{0.2cm}
            \insertnote
        \end{column}
    \end{columns}
}

% --- NOWOCZESNY STYL ---
\usetheme{Antibes}
\usecolortheme{seahorse}
\usefonttheme{professionalfonts}

\setbeamertemplate{blocks}[rounded][shadow=false]
\setbeamercovered{transparent}

% Własna stopka z numeracją
\addtobeamertemplate{navigation symbols}{}{%
    \usebeamerfont{footline}%
    \usebeamercolor[fg]{footline}%
    \hspace{1em}%
    \insertframenumber/\inserttotalframenumber
}

\definecolor{gamebg}{HTML}{282A36}    % Ciemne tło
\definecolor{snakebody}{HTML}{50FA7B} % Neonowy zielony
\definecolor{snakehead}{HTML}{00FF00} % Jaśniejszy zielony
\definecolor{applecolor}{HTML}{FF5555}% Neonowy czerwony
\definecolor{gridcolor}{HTML}{44475A}



% Konfiguracja wyglądu kodu (C++/Python style)
\definecolor{codegreen}{rgb}{0,0.6,0}
\definecolor{codegray}{rgb}{0.5,0.5,0.5}
\definecolor{codepurple}{rgb}{0.58,0,0.82}
\definecolor{backcolour}{rgb}{0.95,0.95,0.92}

\lstdefinestyle{mystyle}{
    backgroundcolor=\color{backcolour},
    commentstyle=\color{codegreen},
    keywordstyle=\color{magenta},
    numberstyle=\tiny\color{codegray},
    stringstyle=\color{codepurple},
    basicstyle=\ttfamily\footnotesize, % Czcionka maszynowa
    breakatwhitespace=false,
    breaklines=true,
    captionpos=b,
    keepspaces=true,
    numbers=left,
    numbersep=5pt,
    showspaces=false,
    showstringspaces=false,
    showtabs=false,
    tabsize=2
}
\lstset{style=mystyle}

\newcommand{\SnakeX}[1]{%
    \ifcase#1 1\or 2\or 3\or 4\or 5\or 6\or 6\or 6\or 6\or 5\or 4\or 3\or 2\or 1\or 1\or 1\or 2\or 3\or 4\or 5\or 6\or 7\or 8\or 9\else 9\fi%
}
\newcommand{\SnakeY}[1]{%
    \ifcase#1 1\or 1\or 1\or 1\or 1\or 1\or 2\or 3\or 4\or 4\or 4\or 4\or 4\or 4\or 5\or 6\or 6\or 6\or 6\or 6\or 6\or 6\or 6\or 6\else 6\fi%
}
% --- DANE PREZENTACJI ---
\title{Prezentacja Beamer}
\author{Dawid Szoka 89221}
\date{\today}

\begin{document}

    \begin{frame}<0>[noframenumbering,label=sekretny_slajd]{Slajd 5: Ukryty (Zdefiniowany na początku)}
        \centering
        \colorbox{black}{\textcolor{green}{\textbf{SECRET LEVEL}}}

        \vspace{1cm}

        W kodzie źródłowym jestem na samym poczatku pliku!\\
        Ale \texttt{againframe} wyświetlił mnie między slajdami 3 i 4.

        \vspace{1cm}
        \textbf{Kliknij w prawo, aby kontynuować do Slajdu 4.}
        \note{
            \textbf{Informacja techniczna:} \\
            Ten slajd jest zdefiniowany jako \texttt{<0>} (niewidoczny) na początku pliku. \\
            Wyświetla się tutaj tylko dzięki "teleportacji" komendą \texttt{againframe}. \\
        }
    \end{frame}
    \begin{frame}[plain]
        \titlepage
    \end{frame}
    \begin{frame}{Slajd 2}
        Jesteśmy na drugim slajdzie. Kliknij w prawo.
    \end{frame}

    \section{Sekcja Główna}
    \begin{frame}{Slajd 3: Drzewo}
        \centering
        \Huge \textbf{3}
        \normalsize
        \vspace{1cm}

        To jest slajd numer 3.\\
        Gdy klikniesz \textbf{W PRAWO}, przejdziesz do "tajnego" slajdu.
    \end{frame}

    \againframe[noframenumbering]{sekretny_slajd}

    \begin{frame}{Slajd 4}
        \centering
        \Huge \textbf{4}
        \normalsize
        \vspace{1cm}

        To jest slajd numer 4.\\
        Właśnie wróciłeś z ukrytego slajdu, mimo że w kodzie sąsiadujemy ze slajdem 3.
    \end{frame}
    \begin{frame}{Snake: Podwójne karmienie}
        \centering
        \begin{tikzpicture}[scale=0.7]
            % TŁO
            \fill[gamebg] (0,0) rectangle (10,8);
            \draw[step=1, gridcolor, thin] (0,0) grid (10,8);
            \draw[ultra thick, gridcolor!50!white] (0,0) rectangle (10,8);

            % --- PARAMETRY GRY ---
            \def\eatOne{8}  % Czas zjedzenia 1. jabłka
            \def\eatTwo{15} % Czas zjedzenia 2. jabłka
            \def\maxFrames{23}

            \foreach \t in {1,...,\maxFrames} {
                \only<\t>{

                % --- LOGIKA JABŁEK ---

                % JABŁKO 1 (do czasu eatOne)
                    \ifnum\t<\eatOne
                    \shade[ball color=apple1] (6.5, 4.5) circle (0.4);
                    \else
                    % Moment zjedzenia 1
                        \ifnum\t=\eatOne
                        \node[star, star points=5, fill=yellow, draw=red, thick, scale=0.9]
                        at (6.5, 4.5) {\textbf{YUM!}};
                        \else
                        % JABŁKO 2 (między eatOne a eatTwo)
                            \ifnum\t<\eatTwo
                            \shade[ball color=apple2] (1.5, 6.5) circle (0.4);
                            \else
                            % Moment zjedzenia 2
                                \ifnum\t=\eatTwo
                                \node[star, star points=5, fill=white, draw=orange, scale=0.8] at (1.5, 6.5) {BIG!};
                                \else
                                % JABŁKO 3 (po eatTwo - pojawia się np. na 9,2)
                                    \shade[ball color=apple3] (9.5, 2.5) circle (0.4);
                                \fi
                            \fi
                        \fi
                    \fi

                % --- OBLICZANIE DŁUGOŚCI WĘŻA ---
                % Logika:
                % Jeśli t > eatTwo -> Długość 5
                % Jeśli t > eatOne -> Długość 4
                % W przeciwnym razie -> Długość 3
                    \pgfmathsetmacro{\currLen}{\t >= \eatTwo ? 5 : (\t >= \eatOne ? 4 : 3)}
                    \pgfmathsetmacro{\myScore}{%
                        int(\t >= \eatTwo ? 200 : (\t >= \eatOne ? 100 : 0))%
                    }
                    % --- RYSOWANIE CIAŁA ---
                    \foreach \i in {1,...,\currLen} {
                        \pgfmathsetmacro{\tailT}{int(\t-\i)}
                        \ifnum\tailT>0
                        % Kolor ogona zmienia się lekko w zależności od segmentu
                        \fill[snakebody, rounded corners=3pt, opacity=0.9]
                        (\SnakeX{\tailT}, \SnakeY{\tailT}) rectangle ++(1,1);
                        \fi
                    }

                % --- RYSOWANIE GŁOWY ---
                    \fill[snakehead, rounded corners=4pt] (\SnakeX{\t}, \SnakeY{\t}) rectangle ++(1,1);

                    % Oczy
                    \fill[black] (\SnakeX{\t}+0.25, \SnakeY{\t}+0.7) circle (0.1);
                    \fill[black] (\SnakeX{\t}+0.75, \SnakeY{\t}+0.7) circle (0.1);

                    % --- INTERFEJS ---
                    \node[anchor=north west, white] at (0,8) {Frame: \t};
                    \node[anchor=north east, white] at (10,8) {
                        Score: \myScore
                    };
                }
            }
        \end{tikzpicture}
        \note<1-7>{
            \textbf{Krok 1: Inicjalizacja} \\
            Wąż startuje z długością 3 jednostek. \\
        }

        % Faza 2: Zjedzenie pierwszego jabłka (Klatka 8)
        \note<8>{
            \textbf{EVENT: Kolizja z jedzeniem!} \\
            Współrzędne głowy pokrywają się z jabłkiem. \\
            - Flaga wzrostu: TRUE
        }

        % Faza 3: Ruch do drugiego jabłka (Klatki 9-15)
        \note<9-14>{
            \textbf{Krok 2: Wzrost organizmu} \\
            Zauważ, że ogon jest teraz dłuższy (4 jednostki). \\
            Wąż zmierza do kolejnego celu.
        }

        % Faza 4: Zjedzenie drugiego jabłka (Klatka 15)
        \note<15>{
            \textbf{EVENT: Drugie karmienie!} \\
            Kolejny punkt zdobyty. \\
            Wąż osiąga maksymalną przewidzianą długość.
        }

        % Faza 5: Koniec sekwencji (Klatki 17-23)
        \note<16-23>{
            \textbf{Stan stabilny} \\
            Wąż ma teraz 5 jednostek długości. \\
            Pojawia się niebieski cel, ale symulacja dobiega końca.
        }
    \end{frame}
% ----------------------------------------------------------------------
% 3. GRAF (15 elementów, transparentny)
% ----------------------------------------------------------------------
    \section{Struktury danych}
    \begin{frame}{Drzewo Binarne (Animacja Sekwencyjna)}
        \centering
        \begin{tikzpicture}[
            scale=0.52, transform shape,
        % Styl węzłów
            main/.style={
                circle,
                draw=blue!40,
                line width=0.8pt,
                top color=white,
                bottom color=blue!15,
                text=blue!60!black,
                font=\bfseries\scriptsize,
                inner sep=1pt,
                minimum size=0.65cm,
                drop shadow={opacity=0.2}
            },
            connect/.style={
                ->, thin, color=gray!60, >=stealth, shorten >=1pt
            }
        ]
            % --- BLOKADA SKAKANIA (Kluczowe!) ---
            \useasboundingbox (-8.5, -7) rectangle (8.5, 1);


            % --- POZIOM 0 (Klatka 1) ---
            \node<1->[main] (1) at (0, 0) {1};

            % --- POZIOM 1 (Klatka 2) ---
            \foreach \i [count=\n from 0] in {2,3} {
                \node<2->[main] (\i) at (-4 + \n*8, -1.5) {\i};
                \draw<2->[connect] (1) -- (\i);
            }

            % --- POZIOM 2 (Klatki 3 i 4) ---
            \foreach \i [count=\n from 0] in {4,...,7} {
                % Obliczamy numer slajdu: dla i<6 slajd 3, dla i>=6 slajd 4
                \pgfmathsetmacro{\s}{int(3 + (\i>=6))}

                \node<\s->[main] (\i) at (-6 + \n*4, -3.0) {\i};

                \pgfmathsetmacro{\parent}{int(\i/2)}
                \draw<\s->[connect] (\parent) -- (\i);
            }

            % --- POZIOM 3 (Klatki 5 i 6) ---
            \foreach \i [count=\n from 0] in {8,...,15} {
                \pgfmathsetmacro{\s}{int(5 + (\i>=12))}

                \node<\s->[main] (\i) at (-7 + \n*2, -4.5) {\i};

                \pgfmathsetmacro{\parent}{int(\i/2)}
                \draw<\s->[connect] (\parent) -- (\i);
            }

            % --- POZIOM 4 (Klatki 7, 8, 9, 10) ---
            \foreach \i [count=\n from 0] in {16,...,31} {
                \pgfmathsetmacro{\s}{int(7 + floor((\i-16)/4))}

                \node<\s->[main] (\i) at (-7.5 + \n*1, -6.0) {\i};

                \pgfmathsetmacro{\parent}{int(\i/2)}
                \draw<\s->[connect] (\parent) -- (\i);
            }

        \end{tikzpicture}
        \note<1-2>{
            \textbf{Wprowadzenie:} \\
            Prezentujemy strukturę drzewa binarnego zupełnego. \\
            - Węzeł korzenia (Root) o indeksie 1. \\
            - Każdy rodzic posiada co najwyżej dwoje dzieci.
        }


        \note<3-6>{
            \textbf{Wzrost wykładniczy:} \\
            Zwróćcie uwagę na symetrię rozrostu. \\
            - Najpierw wypełniamy lewe poddrzewo. \\
            - Następnie prawe poddrzewo. \\
            Liczba węzłów podwaja się na każdym poziomie ($2^k$).
        }

        \note<7-10>{
            \textbf{Warstwa liści (Leaf Nodes):} \\
            Ostatni poziom zawiera aż 16 elementów. \\
            Łącznie na ekranie mamy $2^5 - 1 = 31$ węzłów. \\
            To pokazuje, jak szybko rośnie złożoność pamięciowa w strukturach drzewiastych.
        }
    \end{frame}


    \section{Analiza matematyczna}
    \begin{frame}{Wizualizacja funkcji tłumionej}
        \begin{columns}
            \begin{column}{0.65\textwidth}
                \centering
                \begin{tikzpicture}
                    \begin{axis}[
                        width=\textwidth, height=0.7\textwidth,
                        axis lines=middle,
                        grid=major,
                        xmin=0, xmax=10, ymin=-1.2, ymax=1.2,
                        xlabel=$x$, ylabel=$f(x)$,
                        samples=100
                    ]
                        \foreach \i in {1,...,10} {
                            \only<\i->{
                                \addplot[
                                    domain=(\i-1):\i,
                                    thick,
                                    blue!80!black
                                ] {exp(-0.2*x) * cos(deg(2*x))};
                            }
                        }
                    \end{axis}
                \end{tikzpicture}
            \end{column}

            \begin{column}{0.3\textwidth}
                \begin{block}{Definicja funkcji}
                    \[ f(x) = e^{-a \cdot x} \cos(\omega x) \]
                \end{block}

                \begin{itemize}
                    \item<1-> $\alert{a = 0.2}$ (tłumienie)
                    \item<1-> $\alert{\omega = 2.0}$ (częstość)
                \end{itemize}

                \only<10->{
                    \begin{exampleblock}{Status}
                        Rysowanie zakończone.
                    \end{exampleblock}
                }
            \end{column}
        \end{columns}

        \note{Prezentacja funkcji harmonicznej z wykładniczym zanikiem amplitudy.}
    \end{frame}


    \appendix

    \begin{frame}[fragile, noframenumbering]{Dodatek Implementacja ruchu (C++)}
        \begin{lstlisting}[language=C++, title={SnakeMovement.cpp}]
void updateSnake(int& headX, int& headY, Direction dir) {
    // Aktualizacja pozycji glowy
    switch(dir) {
        case UP:    headY--; break;
        case DOWN:  headY++; break;
        case LEFT:  headX--; break;
        case RIGHT: headX++; break;
    }
    // Wykrywanie kolizji ze sciana
    if (headX < 0 || headX >= GRID_W || headY < 0) {
        gameOver = true;
        return;
    }
    // Logika ogona (FIFO)
    tail.insert(tail.begin(), {headX, headY});
    if (!ateApple) tail.pop_back();
}
        \end{lstlisting}
        \note{
            \textbf{Szczegóły techniczne kodu:}
            \begin{itemize}
                \item Zwrócić uwagę na referencje \texttt{int\& headX} - modyfikujemy zmienne oryginalne, nie kopie.
            \end{itemize}
        }
    \end{frame}


    \begin{frame}[noframenumbering]{Kontakt}
        \centering
        \Large \textbf{Dziękuję za uwage!} \\
    \end{frame}

\end{document}
